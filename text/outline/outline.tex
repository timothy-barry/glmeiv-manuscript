\documentclass[12pt]{article}
\usepackage{amsfonts}
\usepackage{amsmath}
\usepackage{graphicx} 
\usepackage{float}
\usepackage[caption = false]{subfig}
\usepackage{/Users/timbarry/optionFiles/mymacros}

\begin{document}
\noindent
Tim B.
\begin{center}
\textbf{A generalized errors-in-variables model, with application to single-cell CRISPR screens}
\end{center}

\section{Introduction}
CRISPR is a genome engineering tool that has enabled scientists to precisely edit human and nonhuman genomes, opening the door to new therapies for diseases \cite{Rothgangl2021,Musunuru2021} and accelerating basic biology research \cite{Przybyla2021}. Recently, scientists have paired CRISPR genome engineering with single-cell sequencing \cite{Dixit2016,Datlinger2017}. The resultant assays, known as a ``single-cell CRISPR screens,'' link specific genetic perturbations to changes in cellular phenotypes, illuminating regulatory networks underlying human diseases and other traits \cite{Morris2021a}.

Despite their extraordinary promise, single-cell CRISPR screens present substantial statistical and computational challenges. 

% These assays, called ``single-cell CRISPR screens,'' could 
	 % These assays, called single-cell CRISPR screens, promise to transform our understanding of disease by enabling 
	% CRISPR promises to transform clinical treatment of diseases and basic biology research alike: 
%	\begin{itemize}
	% a biotechnology that enables scientists to precisely and efficiently edit genomes.
	% \item CRISPR technology also can be used to drive basic biological discovery. 
	% \end{itemize}
%	\item \textbf{P2}:  \cite{Przybyla2021}
%	\item \textbf{P3}: Despite their considerable promise, single-cell CRISPR screens pose substantial statistical and computational challenges.
%	\item \textbf{P4}: In this work we make two main contributions.
%	\begin{itemize}
%	\item First, we study the thresholding method from both theoretical and empirical perspectives, characterizing its properties and illuminating the settings in which it is expected to fail.
%	\item Second, we propose 
%	\end{itemize}
	
	% \item In summary we couple a novel statistical method -- GLM-EIV -- to powerful technologies from genomics and computer science -- CRISPR genome engineering, single-cell sequencing, clound computing, and high-performance computing -- to yield new statistical and biological insights.
% \end{itemize}
	
\section{Background and analysis challenges}

\begin{itemize}
\item 
\end{itemize}

\section{Thresholding method}

\begin{itemize}
\item 
\item 
\end{itemize}

\subsection{Theoretical analysis}

\subsection{Empirical analysis}

\section{GLM-EIV}

\section{Simulation studies}

\section{Real data analysis}

\section{Discussion}

\section{Appendix}

% \subsection{Dictionary of symbols}

\subsection{Proofs of theoretical results for thresholding estimator}

\subsection{Derivation of EM algorithm}

\subsection{Derivation of observed information matrix}

\subsection{Implementation using R family objects}

\subsection{Statistical accelerations to GLM-EIV}

\subsection{Additional simulation results}

\bibliographystyle{unsrt}
\bibliography{/Users/timbarry/optionFiles/glmeiv.bib}


\end{document}
